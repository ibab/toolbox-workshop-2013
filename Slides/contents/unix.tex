\section{Unix Shell}
\begin{frame}{Unix Shell}
  \tableofcontents[sectionstyle=show/hide,
                   hideothersubsections]
\end{frame}

\subsection{Dateisystem}
\begin{frame}{Dateisystem}
  \begin{itemize}
    \item \texttt{/} trennt Teile eines Pfads
    \item Das gesamte Dateisystem bildet \emph{einen} Baum
    \item Es gibt immer ein aktuelles Verzeichnis
    \item Pfade können absolut oder relativ angegeben werden
    \item drei spezielle Verzeichnisse:
      \begin{itemize}
        \item \texttt{.} das aktuelle Verzeichnis
        \item \texttt{..} das Oberverzeichnis des aktuellen Verzeichnisses
        \item \texttt{\textasciitilde} das Heimverzeichnis
      \end{itemize}
    \item Dateien, die mit \texttt{.} anfangen sind versteckt
  \end{itemize}
\end{frame}

\subsection{Befehle}
\begin{frame}{\texttt{man}, \texttt{pwd}, \texttt{cd}}
  \begin{tabular}{lp{20em}}
    \texttt{man \textit{topic}} & für manual: zeigt die Hilfe für ein Programm \\
    \texttt{pwd} & für print working directory: zeigt das aktuelle Verzeichnis \\
    \texttt{cd \textit{directory}} & für change directory: wechselt in das angegebene Verzeichnis
  \end{tabular}
\end{frame}

\begin{frame}{ls}
  \begin{tabular}{lp{20em}}
    \texttt{ls [\textit{directory}]} & für list: zeigt den Inhalt eines Verzeichnisses an \\
    \texttt{ls -l} & zeigt mehr Informationen über Dateien und Verzeichnisse \\
    \texttt{ls -a} & zeigt auch versteckte Dateien
  \end{tabular}
\end{frame}

\subsection{Nützliche Shell Features}
\begin{frame}{Tastaturkürzel}
  \begin{tabular}{lp{20em}}
    \texttt{Ctrl-C} & beendet das laufende Programm \\
    \texttt{Ctrl-D} & EOF (end of file) eingeben, kann Programme beenden \\
    \texttt{Ctrl-L} & leert den Bildschirm
  \end{tabular}
\end{frame}

\begin{frame}{Globbing (\texttt{*})}
  \begin{tabular}{lp{20em}}
    \texttt{*} & wird ersetzt durch alle passenden Dateien
  \end{tabular}
\end{frame}
